\section{Glossaire}


Voici quelques d�finitions \footnote{Le vocabulaire technique est utilis� en anglais, par souci de coh�rence avec
le code.} techniques:
\begin{description}
\item[Inode]{: Structure des syst�mes de fichiers linux qui sert � garder les informations administratives sur un fichier (taille, nom...)}
\item[Block]{: zone d�limit�e sur le disque contenant un certain nombre
    d'octets (sous ext3, 1024, 2048 ou 4096).}
\item[Data block]{: block destin� sp�cifiquement � contenir les donn�es de
    l'utilisateur (fichiers, ACLs...), plut�t que les structures du syst�me de
    fichiers.}
\item[Block group]{: Groupe de blocks, contenant les structures du syst�me de
    fichier (superblock, inodes, data blocks...). Leur nombre est fonction de
    la taille des blocks et de la taille de la partition.}
\item[Superblock]{:structure des syst�mes de fichiers ext2/3, qui conserve les
    informations sur le syst�me de fichiers lui-m�me (taille d'un block,
    fonctionnalit�s pr�sentes...)}
\item[Group descriptor]{:structure des syst�mes de fichiers ext2/3, qui
    conserve les informations sur le groupe (position de l'inode bitmap,
    nombre de dossiers...)}
\item[Inode table]{: ensemble de tous les inodes d'un block group.}
\item[Bitmap]{: 'carte', qui fait correspondre � un bit l'�tat bool�en d'une donn�e. Dans le cas de l'inode et du block bitmap, il s'agit de savoir si le block est libre (0) ou occup� (1).}
\item[Journal]{: 'partie' d'un syst�me de fichier qui aide � pr�server l'int�grit� de celui-ci en enregistrant ses modifications.}
\item[POSIX]{: Portable Operating System Interface  Ensemble de standards, cr��s pour rendre compatibles entre elles diff�rentes variantes de UNIX (Solaris, Cygwin, Mac OS X, Linux (partiellement)). }
\item[ACL]{: Access  Control List. Listes de permissions d�di�es � la gestion
plus 'fine' des droits utilisateurs (lecture, �criture, execution...), et
disponibles sur plusieurs syst�mes de fichiers (ext3, NTFS...)}
\item[API]{:Application Programming Interface. Biblioth�que contenant des
    fonctions pouvant �tre appel�es depuis un programme.}
\end{description}